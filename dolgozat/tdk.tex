\documentclass[a4paper,oneside]{book}

% Egy hasznos Mac OS X-es videó:
% http://www.youtube.com/watch?v=XFiLjB9UNaA&feature=related

% Oyepa, Working Tagged Filesystem:
% http://pages.stern.nyu.edu/~marriaga/software/oyepa/

\usepackage{tdk}

\usepackage{listings}
\usepackage{tikz}

\usepackage{rotating}


\usepackage{color}
\usepackage{soulutf8}
% \definecolor{comment}{rgb}{0,0.6,0.2}

\usepackage{CJKutf8}
\usepackage{graphicx}
\usepackage[utf8]{inputenc}
\usepackage{hyperref}
\usepackage{marginnote}
\usepackage{listings}
\graphicspath{ {../dolgozat/images/} }


% Code colorature
\definecolor{paszt}{RGB}{252,252,252}
\definecolor{keret}{RGB}{220,220,220}
\lstset{
backgroundcolor=\color{paszt},
% showlines=true,
framexleftmargin=4mm,
framexrightmargin=4mm,
framextopmargin=2mm,
framexbottommargin=2mm,
frameround=tttt,
frame=trbl,
rulecolor=\color{keret},
extendedchars=true,
literate={á}{{\'a}}1 {é}{{\'e}}1 {ö}{{\"o}}1 {ő}{{\H{o}}}1 {ó}{{\'o}}1,
%inputencoding=utf8
}

% Line spacing (1.5-ös sorköz)
\linespread{1.2}

% Set spacing in source code (normál sorköz -1.5)
%\lstset{
%lineskip={-3pt}
%}

\usepackage{pslatex}

%%% Font settings
% \usepackage{times}
\usepackage{lmodern} % !!!
% \usepackage{chancery}
% \usepackage{charter}
% \usepackage{palatino}
% \usepackage{mathpazo}
% \usepackage{mathpple} % !!!
% \usepackage{mathptmx}

\usepackage{pstricks}

\usepackage{cpp}
\usepackage{python}

% Indicators
\usepackage{indicators}

\begin{document}

% \pagestyle{empty} %a címlapon ne legyen semmi=empty, azaz nincs fejléc és lábléc

% Címlap
\include{cimlap}

% Font size
% \large

% Tartalomjegyzék
\tableofcontents
\thispagestyle{empty}
\cleardoublepage

% Set page style !!!
\pagestyle{fancy}

% Set number of page
\setcounter{page}{1}

% Chapters
\Chapter{Bevezetés}

A Mesterséges Intelligencia (MI) kutatás napjaink egyik legdinamikusabban fejlődő és egyben legtöbbet vitatott tudományterülete. A dinamikus fejlődés lendületét pedig minden esetben egy új gépi tanulási szakterület, a deep learning, és a hozzá kapcsolódó neurális hálózatokra épülő módszerei adják.

A neurális hálózatokat az adatokból való tanulás, az optimalizálási problémák megoldásának képessége vagy az asszociatív képesség különösen alkalmassá teszi a különböző felismerési feladatok, speciálisan a képfelismerési feladatok megoldására, illetve egyéb képfeldolgozási műveletek elvégzésére.

A képfeldolgozás és alakzatfelismerés területén az úgynevezett konvolúciós neurális hálók (convolutional neural network -- CNN) megjelenése hozott áttörést. A módszer alkalmas például arcfelismerése, ábrák feliratozására vagy például biztonsági kamerák felvételeinek automatikus feldolgozására és kiértékelésére is. A konvolúciós hálók áttörő sikere miatt mára széles körben elterjedt a használata.

A dolgozat fő témája a kínai karakterek felismerése. A kutatás bemutatása során kitérek arra, hogy hogyan generálhatunk és adhatunk hozzá különböző zajokat a képekhez.

Részletezem a CNN müködését és előnyeit. Feladatom során megemlítek már létező optikai karakterfelismerő (OCR) rendszereket. Kitérek azok működésére és felépítésére.

Továbbá bemutatom hogyan működik a tanítás és a tesztelés fázis. Részletezen, hogy hogyan zajlik a tanulás a neurális hálózatokban.

\Chapter{Kínai karakterek felismerése}

{\large \textbf{Az alapvonások}}

Az írásjegyek felépítésének következő lényeges szabálya az írásjegy vonásainak sorrendje. Az írásjegyek – bármilyen bonyolult legyen is némelyik – tulajdonképpen néhány igen egyszerű vonalból épülnek fel. Ezek az írásjegyek alapelemei, vagy alap-ecsetvonásai. Az alábbi képen az alapvonások néhány főbb típusa látható. Természetesen az alapvonásoknak több változata is lehetséges (méret, vastagság, irány) attól függően, hogy az írásjegy melyik részén helyezkedik el.

\begin{center}
	\includegraphics[width=0.4\linewidth]{images/chinese_strokes.png}
\end{center}

Minden egyes vonásnak megvan a felépítési szabálya: az ecsetvonásoknak meghatározott sorrendben kell követniük egymást, még pedig általános elvként az írásjegyek határait alkotó virtuális négyszög bal felső sarkából lefelé és jobbra haladva. Az írásjegy gerincét, fő szerkezeti elemét adó nagyobb vonást, ha az egész írásjegyet átjárja, legutoljára húzzák.

\newpage
{\large \textbf{A vonássorrend szabályai: }}
\begin{enumerate}
	\item A vízszintes vonások megelőzik a függőleges vonásokat.
	\item A balra lejtő vonások megelőzik a jobbra lejtő vonásokat. 
	\item Az írásjegyek írását felülről kell kezdeni. 
	\item Az írásjegyet balról jobbra haladva építik fel. 
	\item A felülről keretezett írásjegyeknél előbb a keretet kell meghúzni. 
	\item Az alulról keretezett írásjegyeknél a keretet legvégül kell meghúzni. 
	\item A teljes keretet mindig legvégül kell bezárni.
\end{enumerate}

Egy szimmetrikus felépítésű írásjegynél előbb a középső részt kell kialakítani, s csak azután az oldalakat.

\begin{center}
	\includegraphics[scale=1.0]{images/vonasrend_ordered.png}
\end{center}

A kínai írásjegyek különböző számú alapvonásokból épülhetnek fel. Ezek közül a legegyszerűbb a csupán egyetlen vízszintes vonalból álló „egy” jelentésű \begin{CJK*}{UTF8}{gbsn}
一
\end{CJK*} ji írásjegy. A kínai írásrendszer más, egy vonásból álló írásjegyet nem tartalmaz. Aránylag ritkák a két vonásból álló írásjegyek is, például: \begin{CJK*}{UTF8}{gbsn}
二
\end{CJK*} er„kettő”,
\begin{CJK*}{UTF8}{gbsn}
十
\end{CJK*} si „tíz”,
\begin{CJK*}{UTF8}{gbsn}
人
\end{CJK*} zsen „ember” stb. A hagyományos írásjegyek zöme 15–30 vonásból épül fel (átlagosan 9 vonásból). Esetenként azonban ennél jóval több vonásból álló írásjegyek is előfordulhatnak, melyek tulajdonképpen már több önálló írásjegy összevonásának is tekinthetők. Ritkák ugyan, de léteznek 50 vagy akár 80 vonásból álló írásjegyek is.\\

{\Large OCR megvalósítások}\\

Az optikai karakterfelismerés feladata\\

A különböző formátumú dokumentumok kezelésének egyik speciális esete, amikor a kezelendő dokumentumok még nem állnak rendelkezésre elektronikus formában. Ebben az esetben szinte mindig arról van szó, hogy a dokumentumok kinyomtatva, papír alapú hordozón jelennek meg. Szövegbányászati tevékenység végzéséhez értelemszerűen digitalizálni kell a még nem digitalizált, papíron, nyomtatásban vagy írásban meglévő dokumentumokat, azaz a képként érzékelt dokumentumot szövegfájl formátumba kell átalakítani, hogy abban az után elektronikusan szerkeszthető és feldolgozható legyen. Ebben a szituációban kap szerepet az optikai karakterfelismerés (optical character recognition, OCR), amely így szövegbányászati előfeldolgozásnak tekinthető. Az optikai karakterfelismerés a mesterséges intelligencia jelfeldolgozó és generalizációs képességeit kiaknázva képes magas hatékonysággal nyomtatott, papír alapú dokumentumokon lévő karaktereket felismerni.

Az alap probléma itt az, hogy a nyomtatott papír alapú dokumentumok esetében nagy zajaránnyal kell megküzdeni annak érdekében, hogy a releváns információt kihámozzuk az érzékelt képi jelek és minták közül. Nyomtatott dokumentum esetében ilyen zajnak tekinthető például egy apró folt a papíron, tintaelmosódás, tintahiány, homályos háttér, apró gyűrodés a papíron, túl közeli vagy egybeolvadó betűk, betű dőlésszögének ingadozása stb. Kézírás esetén a kihívás még nagyobb, hiszen itt a személyiségjegyek sokszínűségéből adódó írásminták kavalkádjából kell kihámozni a karaktereket. Mind a nyomtatott, mind pedig a kézírásos esetben az optikai karakterfelismerő rendszer egy tanulási fázist követően képes olyanmintákat is osztályozni (a megfelelő karaktert felismerni), amelyekkel a tanulási fázisban nem találkozott, tehát megvan a szükséges generalizációs képessége.

Az OCR-rendszerek alapvetően két részből állnak: egyrészt a szkennelő fejből, amely a dokumentum egészét vagy részeit beszkenneli, másrészt pedig magából a mesterséges intelligencia szoftverből, ami elvégzi a beérkezett minták osztályozását, azaz magát a karakterfelismerést.

\begin{center}
\includegraphics[scale=0.65]{ocr}
\end{center}

Szegmentáció

A szegmentáció során a karakterek közötti éles határ megtalálása a cél annak érdekében, hogy téves minták ne kerüljenek osztályozásra (pl. két fél karakter). A szegmentáció feladata lehet az is, hogy a karakter-dőlésszögeket, karakterméreteket normalizálja. Sok esetben a szöveges dokumentumokban nem csak karakterek vannak, hanem képek és egyéb, a felismerés szempontjából nem lényeges szimbólumok. A szegmentáció további feladata tehát az is, hogy az ilyen, számunkra nem releváns grafikus objektumok közül kiszűrje a csak karaktereket tartalmazó szöveges részeket.

\begin{center}
\includegraphics[scale=1.0]{ocr_segmentation}
\end{center}

Optikai előfeldolgozás

Az előfeldolgozás a bemeneti minta komplexitásának csökkentésére szolgál, és annak legjellemzőbb vonásait elemi ki. Különösen nagy jelentősége van a kézírás felismerésekor, ugyanis az írott betűk jóval komplexebb mintákat alkothatnak, mint a nyomtatott betűk. A jellemzőkiemelés során a komplexitás úgy csökken, hogy közben a legjellemzőbb információk megmaradnak és ezáltal a későbbi feldolgozás számításigényét redukálhatjuk. Ez a folyamat tulajdonképpen egy komplexitáscsökkentéssel járó digitalizáció. Az alábbi ábra egy egyszerű digitalizálási módszert mutat, amikor az analóg jelre egy mátrixot reprezentáló rácshálót illesztünk, és amelyik cellán átmegy az analóg karakter, az az elem a mátrixban 1 értéket vesz fel (fekete), egyéb esetben pedig 0-t (fehér).

\begin{center}
\includegraphics[scale=0.65]{ocr_preprocess}
\end{center}

Felismerés

Az osztályozás során történik meg a tényleges karakterfelismerés. A karakterfelismerő módszer a bemeneti jellemzővektor alapján dönti el, hogy az ismert karakterek közül melyikre hasonlít a legjobban a bemeneti vektor. Így a karakterfelismerési probléma egy asszociatív memóriát igénylő feladat, amelynek során a tárolt memóriaelemek közül kell előhívni azt, amely a bemeneti mintának legjobban megfelel.\\

Számjegyek felismerése zajos képeken\\

A következőkben bemutatok néhány olyan alkalmazást, amelyek az optikai karakterfelismerés egyes részeit hivatottak megvalósítani. Ezek között szerepelnek olyanok, amelyek meglehetősen szűk témakörrel, például számjegyek felismerésével foglalkoznak, de lesznek olyanok is, amelyek meglehetősen komoly eredményeket képesek felmutatni a kézírásos karakterek felismerése terén. Minden esetben részletesen kitérek arra, hogy az alkalmazás készítése során milyen megfontolások vezettek az adott neurális hálózat felépítésének megválasztásához, bemutatom az egyes területeken megjelenő és kezelendő problémákat, illetve összehasonlításokkal és adatokkal támasztom alá az egyes rendszerek hatékonyságát és alkalmazhatóságát a kijelölt problémára. 

A zajos számjegyek és karakterek a rendszámtáblák felismerésénél általános jelenségnek számítanak, mivel itt a kamera felvétele általában mozgó járműről készül, és a zajt tovább növelheti a piszok, a nedves, esetleg esős időjárás, éjszakai sötétség, a tükröződés stb.

Jelen esetben tekintsünk el a különböző felvételi körülmények által okozott természetes hibáktól és vizsgáljuk meg, milyen eredménnyel ismerünk fel mesterségesen eltorzított képeken számjegyeket. A beolvasott képeket a következő módszerekkel tesszük zajosabbá: 

\begin{itemize}
\item Gauss zajok alkalmazásával. 
\item Képfeldolgozási eszközök használatával (elmosás). 
\item Véletlenszerűen részletek kitörlésével. 
\end{itemize}

Mivel a program hatékonyságát nagymértékben befolyásolja a neurális hálózat felépítése, gondosan meg kell tervezni, hogy milyen hálózatot alkalmazunk az egyes problémák megoldásához. Jelen esetben az átlagos négyzetes hiba (MSE) segítségével értékeljük a hálózat hatékonyságát és több hálózat teljesítményét összevetve választjuk ki, hogy milyen felépítést használunk a jelenlegi problémánkhoz. 

Első lépésben ki kell választanunk a rejtett és a kimeneti réteg neuronjai által alkalmazott aktivációs függvényt. Jelen esetben a rejtett rétegbeli neuronoknál tangens szigmoid, a kimeneti réteg neuronjainál pedig logaritmus szigmoid függvényt alkalmazunk. 

Természetesen lehetőségünk lenne akár minden rejtett rétegbeli neuronnál, illetve minden kimeneti neuronnál ugyanazt vagy akár mindegyiknél különböző aktivációs függvényt használni, de jelen esetben, a négyzetes hiba mértékét figyelembe véve ez a két függvény megfelelőnek látszik arra, hogy a segítségével a neurális hálózatunk nagy hatékonysággal ismerje fel az egyes számjegyeket.

Szintén próbálgatásos módszer segítségével a rejtett neuronok számát 9-ben határoztuk meg. Rögzített 50 tanítási epoch mellett a legkisebb négyzetes hibát adó rejtett rétegbeli neuronszámot választottuk.

A következő táblázatból látszik, hogy nem lehet egyértelműen megmondani, hogy az egyes neuronszámok mellett milyen hibaértékek várhatók. A táblázatból az is kitűnik, hogy 9 rejtett neuronnal a négyzetes hibát látványosan kisebbre lehetett visszaszorítani. 

\begin{center}
\begin{tabular}{ |c|c|c| } 
 \hline
 Rejtett neuronok száma  & MSE  \\ 
 \hline\hline
 1 & 0,1424 \\
 \hline
 2 & 0,0485 \\
 \hline
 3 & 0,0508 \\
 \hline
 4 & 0,0745 \\
 \hline 
 5 & 0,0809 \\
 \hline
 6 & 0,0933 \\
 \hline
 7 & 0,1183 \\
 \hline 
 8 & 0,0386 \\
 \hline
 9 & $1,9509*10^{-12}$ \\
 \hline
 10 & 0,0039 \\
 \hline
 11 & 0,1691 \\
 \hline
 12 & 0,0025 \\
 \hline
 13 & 0,0014 \\
 \hline
 14 & 0,0014 \\
 \hline
\end{tabular}
\end{center} 

A neurális hálózat ideális felépítésének meghatározása után a következő lépés a tanulóhalmaz összeállítása. A számjegyek felismerésére készített hálózatot a következő tanulópéldákkal tanítjuk be: 

\begin{itemize}
\item Normál számok 
\item Háromszor és tizenkétszer elmosott számok 
\item Öt és harminc közötti intenzitású Gauss zajjal módosított számok
\end{itemize}

A normál számokat 100\%-ban, az elmosott számokat átlagosan 95\%-ban, a Gauss zajjal kezelt számokat átlagosan 94\%-ban ismerte fel.

Nyomtatott karakterkészlet

Ebben a részben két speciális betűtípussal nyomtatott karakterek felismerését tekintjük át egy neurális hálózatot alkalmazó program működésén keresztül. Mindkét betűtípus 84 karaktert tartalmaz, a karakterek pedig 8 x 8-as felbontásúak.

A számjegyek felismerésének problémája látványosan egyszerűbb, mint az általános karakterfelismerés által támasztott feladat. Első lépésként vizsgáljuk meg, hogyan és milyen hatékonysággal ismerhetünk fel nyomatott karaktereket.

A feladat megoldásához használt hálózatban 64 bemeneti neuront alkalmazunk. A hálózat általános előrecsatolt neurális háló. A 64 bemeneti neuron mindegyike a 8 x 8-as karakter egy-egy pixeléhez van kötve. A bemenet fekete pixelek esetén 1, egyéb esetekben 0 értékű.

Mivel a neurális hálózatok hatékonysága nagymértékben függ a hálózat méreteitől, elmondhatjuk, hogy a fenti módszerrel felépített hálózattól nem várhatunk túlságosan hatékony működést. Természetesen egy ilyen hálózat is betanítható oly módon, hogy szembetűnő eredményeket mutasson fel, de ehhez lényegesen több erőforrásra és időre van szüksége, mint az előző példában bemutatott, jóval kisebb hálózatnak.

A rendszer teljesítményének meghatározásához először külön-külön, aztán összesítve is megvizsgáljuk a két választott karakterkészletből vett minták felismerési arányát. Az értékelésnél külön vesszük a rendszer által ismeretlennek titulált és a rosszul felismert karaktereket.

A két karakterkészlet összesített eredményeiből levonhatjuk azt a következtetést, hogy egy neurális hálózat számára valóban jóval nagyobb nehézséget jelent, ha nem „tudja”, hogy pontosan milyen karakterekre számítson. Ez a feladat már előrevetíti, hogy milyen nehézségekkel kell szembenéznünk, ha nem egy fix karakterkészletből választott betűket akarunk a rendszerünkkel felismertetni, mint ahogy ez a kézzel írt karakterek esetében is történik. Mindemellett a különböző betűkészletből vett nyomtatott karakterek felismerése annyiban tűnhet egyszerűbbnek, hogy ha fix számú betűtípust használunk, akkor még mindig tekinthetjük úgy, hogy fix karakterkészlettel van dolgunk és elégséges a hálózatnak minden betűtípusból egy-egy mintát megtanítani az egyes betűkből.

\begin{center}
\begin{tabular}{ |c|c|c| }
\hline
\multicolumn{3}{|c|}{\textbf{Összesített eredmények}}\\
\hline
Felismert karakterek & 99 db & 54\%\\
\hline
Ismeretlen karakterek & 5 db & 3\%\\
\hline
Rosszul felismert karakterek & 64 db & 38\%\\
\hline
Összes ismeretlen vagy rosszul felismert karakter & 69 db & 41\%\\
\hline
\end{tabular}
\end{center}


Kézzel írt karakterkészlet

Az előbbi két példából látszik, hogy a számjegyek felismeréséhez képest a tetszőleges nyomtatott karakterek felismerése jóval nehezebb feladat még abban az esetben is, ha csak limitált számú, megadott betűtípusokból vesszük a bemeneti adatokat. A kézírásos karakterek felismerése még ennél a pontnál is jóval tovább megy, hiszen gyakorlatilag nem támaszkodhatunk arra, hogy egy adott karakterkészletből kell kiválasztanunk a megfelelő betűt.

A kézírások változatosságából adódóan nagyon körültekintően kell megválasztanunk, hogy milyen módszert, illetve milyen felépítésű neurális hálózatot használunk, hiszen nem csak a felismerés hatékonyságát, hanem a rendszer betanítási és futási idejét is figyelembe kell
vennünk.

Ebben a részben egy olyan neurális háló alkalmazással ismerkedünk meg, amely már kézzel írt karaktereket kap bemenetként és ezek felismerésére tesz kísérletet. 

A kézzel írt karakterek felismerésének első lépéseként a felismerni kívánt karaktereket szeparálni kell a képen található többi információtól, illetve a szavakat, szókapcsolatokat karakterekre kell bontani. A kép ilyen formájú feldolgozása után az egyes karaktereket tartalmazó képeket digitalizálni kell. A digitalizálás folyamán a képekből egy bináris mátrix készül. Minden fekete pixelnek egyest, minden fehérnek nullát feleltetünk meg. 

A következő ábrán látható, hogy milyen lépések folyamán jutunk el a kézzel írt A betűtől addig a bitmátrixig (I), amelyet a neurális hálózat már értelmes bemenetként képes kezelni. A digitalizálás folyamán minden karaktert ugyanakkora méretűvé alakítunk, így a hálózat minden esetben egy előre meghatározott méretű bitmátrixszal dolgozik majd. 

\begin{center}
	\includegraphics[scale=0.75]{HandwrittenA}
\end{center}

A karaktereket ennél a programnál felügyelt tanulással tanítjuk be a rendszernek. Minden karakterhez tartozik egy \textit{címke}, amely azt határozza meg, hogy az adott kép milyen betűt reprezentál. Minden címkéhez számos karaktert használunk fel a tanítási folyamat során, így a rendszer többféle variációban is "látja" az egyes betűket. A tanításhoz egy \textit{M} bemeneti mátrixot használunk.

A \textit{k}-adik megtanulandó karakterhez tartozó súlymátrixot \(W_k\)-nak nevezzük. Ez a mátrix alapértelmezés szerint nullmátrixként van inicializálva. A súlymátrix változtatására egy egyszerű algoritmust használunk.

\begin{lstlisting}[language=Python]
for i in range(1, x):
	for j in range(1, y):
		W[i][j] = W[i][j] + M[i][j]
\end{lstlisting}

Ezt az algoritmust minden egyes tanulópélda esetén lefuttatjuk, ezáltal kialakítjuk a neurális hálózatban a felismeréshez használt súlyokat.

Példaként tekintsük meg az S betű három eltérő előfordulását és vizsgáljuk meg az S betűhöz tartozó súlyvektort a tanítási fázis után.

\begin{center}
\includegraphics[scale=0.75]{ocr_S}
\end{center}

Az ábrán látható, hogy az S betűt többféle módon is le lehet írni. Ezek közt a betűk közt az emberi agy ugyan érzékeli a különbséget, azonban elmondható, hogy még ennél sokkal nagyobb eltérések esetén is képesek lennénk felismerni egy S betűt.

Ha jobban megfigyeljük a betűket, észrevehetjük azokat a szemmel látható hasonlóságokat, amelyek alapján az agyunk megállapítja, hogy ezek a képek S betűket ábrázolnak. Ha egy S betűt magunk elé képzelünk, mindenképpen egy jobb felső sarokból induló görbét látunk, amely valahol középen irányt változtat és a bal alsó sarokban fejeződik be. Az ábrán látható három S betű magán viseli ezeket a jellegzetességeket, habár ezeket konkrét reprezentáció esetén csak nehezen tudjuk leírni. A korábban leírt tanítási algoritmus során a súlyvektorban kiemeljük azokat a részeket, amelyek leghangsúlyosabbak az egyes karakterek előfordulásainál.

A rendszerben használt hálózat bemenete egy bitmátrix (tanulópéldák esetén M). A bemenetet n darab különböző karakter felismerésének tanítása esetén n darab neuronnak adjuk át.

A vizsgált S betű egyik megtanult S mintával sem mutat teljes egyezőséget, a hálózat mégis 0,68-as felismerési hányadost számolt ki hozzá, amelynél már megállapíthatjuk az egyezést.

Ha ugyanennyi tudással rendelkező rendszerünknek egy másik betűt, mondjuk egy P-t adunk meg bemenetként, abban az esetben a felismerési hányados jóval kisebb (0,21) lesz, tehát a csupán S betűket ismerő rendszer biztosan nem fogja felismerni.

A rendszer tudásbázisa könnyedén bővíthető új karakterek, illetve a már megtanult karakterek újabb változatainak megtanításával. Maga a rendszer meglehetősen általános és a karakterek méretére és arányaira invariáns.
\Chapter{Minták generálása}

Tanító mintapontok előállítása

A megvalósító neurális hálózat betanítása felügyelt tanulás módszerrel történik. A minták alapján történő tanulás lényege, hogy az eljárás során a be és kimeneti mintapárokból igyekszünk megfelelő ismereteket kinyerni és ezzel a rendszer viselkedését módosítani. A hálózat feladata, hogy megtanulja a rendelkezésre álló mintapont párok által reprezentált bemenet-kimenet leképezést. Ehhez elő kell állítani a megfelelő adathalmazt.

Az adathalmaz előállítása elött definiálni kell a rajzolás dinamikáját. A vonal kirajzolás sorrendje mellet fontos a vonal vastagsága is. Az kézírás ugyan sokat változott, de az alapjai megmaradtak. A stroke-ok vastagsága az ecset gyorsaságától, az ecsetre ható nyomás nagyságától függ. A megfelelően rajzolt vonalakat nagyon fontosnak tartják a kínaiak, mivel a kultúrájukhoz tartozik. A kalligráfia elárulhatja az ember nemét, korát, személyességét és így tovább.

\begin{center}
\includegraphics[scale=0.8]{calligraphy}
\end{center}

A következőkben részletezném a vonal vastagság változását a stroke-ok rajzolásának függvényébe. 

\begin{itemize}
\item A stroke-ok vége elvékonyul, ezt a valóságban az ecset hirtelen felemelésével érik el. (\textit{Van néhány kivétel: right fally} \includegraphics[scale=1.0]{right_fally}) 
\item Ha egy stroke végéből kezdődik egy másik stroke, akkor azok találkozásánál a vonalvastagság növekszik.
\item A horizontális vonalak közepe elvékonyul majd a végén újra vastagabb lesz. A vastagságot súlyokkal könnyedén lehet definiálni.

\begin{center}
\includegraphics[scale=0.6]{horizontal_line}
\end{center}

\item A kampós vonalaknál (hooked stroke) a "kampó" hirtelen vékonyodással és irányváltoztatással jár. Az ecset hirtelen felemelésével érik el. Általában a kampó \includegraphics[scale=1.0]{hook} stroke része.
\item A további stroke-ok általában állandó vonal vastagsággal (ecset gyorsaság állandó). A kézzel írás és festés egyén függő.
\end{itemize}

A vastagság implementálásának módjai
\begin{itemize}
\item Súlyok 

\item Sebesség 

\item Szín átmenet 


\end{itemize}

Környezet
\begin{itemize}
\item Papír -> Képernyő
\item Ecset -> Ellipszis és azokat összekötő poligonok (OpenCV példa)
\item Karakter kombinációk (stroke sorrend) -> Lehet számozással, tömben deklarálva, mátrix térkép szerint stb..
\item Elrendezés, karakterek közötti távolság -> Viszony párokkal, láncolt lista (stroke párokkal).
\end{itemize}

\textit{Line break}\\\\

\begin{itemize}
\item Ecset dinamika
	\begin{itemize}
	\item \(P_1 = (x_1, y_1, d_{x_1}, d_{y_1}, s_1)\) \(P_2 = (x_2, y_2, d_{x_2}, d_{y_2}, s_2)\)
	\item Ecset szín (átmenetesség)
	\end{itemize}
\item Karakter kirajzolás
	\begin{itemize}
	\item Poligonos közelítés
	\item Procedurális rajzolás
	\item Pontonkénti színszámítás (fekete-szürke árnyalat-fehér)
	\item \textit{Görbék kirajzolása (Hermit)}
	\end{itemize}
\item Ideális karakter megjelenítés
\item Ideális karakter zajosítássa
	\begin{itemize}
	\item Különböző zajok
	\item Hálózat robosztussága
	\end{itemize}
\end{itemize}

\begin{comment}{Dolgozok a Hermit-es részen.}
\end{comment}

\begin{comment}{Ezek így vázlatnak jók, viszont az egyes pontokból legalább külön-külön szakaszoknak kellene majd lenni!}
\end{comment}

\Chapter{Neurális hálók}

A neurális hálózattokat gyakran hasonlítjuk az emberi agy működéséhez. Ha körültekintően szemügyre vesszük agyunk működését, akkor azt tapasztaljuk, hogy neuronokból és közöttük felépülő kapcsolatokból áll össze. A külvilágból érkezett ingereket értelmezhetjük úgy, mint egy bemenetet, amit az agyunkban lévő neuronok feldolgoznak. 

\begin{center}
	\includegraphics[scale=1.0]{neuron}
\end{center}

A kutatók az agy felépítését vizsgálva egy olyan matematikai modellt dolgoztak ki, amely   reprezentálni próbálja az agyban található neuronokat és a közöttük lévő kapcsolatokat. Ezt a modellt nevezzük neurális hálónak vagy neurális hálózatnak.  

A neurális hálózatot alkotó neuronok úgynevezett rétegekbe rendeződnek. Háromféle réteget különbözetünk meg, a \textbf{\textit{bemeneti}}, a \textbf{\textit{kimeneti}} és a \textbf{\textit{rejtett réteget}}. Bemeneti és kimeneti rétegből minden hálózatban egy darab van, rejtett rétegből azonban tetszőleges számú lehet. 

A hálózatban a rétegeket élek kötik össze egymással, amelyekhez egyenként egy-egy \textbf{\textit{súly}} tartozik. A neuronok a bemeneti éleiken kapott értékek és a súlyok segítségével bizonyos műveleteket végeznek el, majd az eredmény a kimeneti éleiken keresztül továbbítják a következő réteg neuronjai felé. 

A tanítási folyamat elvégzésekor a hálózatba olyan bemenetet juttatunk, amelyhez tartozó kimenet ismert. A bemenetet végig futtatjuk a hálózat rétegein, majd a kimeneti réteg által szolgáltatott eredményt összehasonlítjuk a kimenet várt értékével. A két érték közötti eltérést a hálózat \textit{\textbf{hibájának}} nevezzük. A tanítás folyamán a hálózat súlyait úgy változtatjuk, hogy ez a hiba lehetőleg minél kisebb legyen. A hálózat betanítása után már olyan bemeneteket is megadhatunk, amelyeknek már nem ismerjük a kimenetét, és a hálózat ezekre is képes hibahatáron belüli kimenetet produkálni.\\

\begin{figure}
	\centering
	\includegraphics[scale=0.75]{ANNLayers}
	\caption{Egy általános felépítésű neurális hálózat}
\end{figure}

\begin{Large}
Perceptron
\end{Large}

Az egyszerűség kedvéért vizsgáljunk meg egy olyan neurális hálózatot, amelynek egyetlen neuronnal rendelkezik. Ezt szokás \textbf{\textit{Perceptron}}-nak is nevezni. 

Részei: 

\textbf{\textit{Bemenet}}: Az kiértékelendő adat (ember számára ingerek), amit általában egy vektor (\(x\)) reprezentál. 

\textit{\textbf{Súlyok}}: Két neuron közötti kapcsolat. Egy valós szám (eleinte véletlenszerűek). A hálózat súlyait (W) mátrixba tároljuk el.  

\textbf{\textit{Összegző csomópont}}: A bemeneteket összeszorozza a megfelelő súlyokkal és ezek összegét képezi. Tulajdonképpen mátrix szorzásról van szó.  

\[v(n) = \sum_i^{n}(w_ix_i)\]

\textit{\textbf{Aktivációs függvény}}: Egy olyan függvény ($\varphi$), ami leképezi a kapott összeget egy kisebb intervallumba pl. [0,1] vagy [-1,1] között. 

\textbf{\textit{Kimenet}}: A leképezett értékünk lesz a kimenetünk (\(y\)). 

\begin{center}
	\includegraphics[scale=0.6]{ANNParts}
\end{center}

\marginnote{\textit{Mi lenne ha az összes generált súly 0 értékű?}}[-0.66cm]

A képen láthatunk egy \(b_n\) változot másnéven bias, amit most nem részleteznék. Alapvetően arra való hogy aktiváljon egy neuront. A \(b(n)\)-hez tartozó súly (\(w_b\)) általában -1 vagy 1.\\


{\Large Backpropagation}

Amennyiben az \(x(n)\) bemenethez tartozó ideális kimenetet \(d(n)\)-nel jelöljük, illetve \(y(n)\) jelenti a hálózat által az \(x(n)\) bemenetre adott kimenetét, a neurális hálózat négyzetes hibáját a következőképpen értelmezzük: 

\[ \varepsilon = (d(n) - y(n))^2\]

Ezt a hibát akarjuk a tanítási eljárás során minimálisra csökkenteni. Természetesen az lenne az ideális, ha a hibát egészen nullára tudnánk redukálni, de ez általában nem sikerül, ezért meg kell elégednünk egy kellően kicsiny hibaküszöbbel. 
\Chapter{Validáció}

\section{Adathalmaz}

A kínai karaktereket különböző kategóriákba is besorolhatjuk az előállításuk szerint.
\begin{itemize}
\item Nyomtatott: A kijelzőkön megjelenített karakterek. Szélességét és magasságát pixelben adjuk meg. A felbontás növelése minőség növekedést eredményez, de a felismerés ideje nő. Több mint 100 betűtípus létezik.
\item Kézzel írt: A betűtípus íronként változó. Léteznek nagy adathalmazok, ami segíti a gépi tanulást a kínai karakter felismerés során.
\item Generált: Ebben az esetben egy modellt kell létrehozni, ami szerint generálunk. Az alapvonások elhelyezkedésének szabályait figyelembe véve lehetőségünk van bizonyos karakterek generálására.

Grafika szempontjából két csoportra gondolhatunk generálás során.
	\begin{itemize}
	\item Raszteres: Minden egyes képkockának megvan a maga színértéke, így áll össze a kép. Ebben az esetben a számítás igény magas (sok képkocka -> sok számítás).
	\item Vektoros: Gyakorlatilag matematikai képletekkel rajzolunk. Egy egyenes (vektor) pontos meghatározásához három adat szükséges: kiindulópont, végpont koordinátái és a vonalvastagság.
	\end{itemize}
\end{itemize}

A generált adathalmazokat használata elönyös, hiszen nincs hozzáférési kötöttség az adatokhoz. Továbbá rugalmas, hiszen a modell változtatásával könnyedén lehet különböző karaktereket létrehozni (akár nem látottakat is). A minták generálása során vektoros grafikát használtam.

Az adatkészletek gray-scaled képek, 255 pixel értékű háttérképen.

\subsection{A mintaadatok előkészítése a tanításhoz}

A tanító mintáknak különböznie kell a tesztelési mintáktól. A tanítási mintahalmazra a hálózat pontossága magasabb, hiszen a tesztelési pontokat nem ismeri.

Az arányok megválasztásánál gyakori a 80/20 (tanító/tesztelő) arány. Az én esetemben 4GB tanító 1GB tesztelő adathalmazt használok.

Az offline adatokkal végzem el a tanítást.\Aref{fig:offline_dataset}. ábrán láthatunk példát a kínai karakterekre.

\begin{figure}[h]
	\centering
	\includegraphics[scale=2.0]{images/offline_dataset}
	\caption{Offline adatbázis példa}
	\label{fig:offline_dataset}
\end{figure} 

Az adathalmaz bejárása előtt azokat érdemes össze keverni véletlenszerűen. Ennek eredménye, hogy a hálózat minden egyes futtatás során egy picivel máshogy fog tanulni. A különbözöségek elkerülésére lehet használni egy úgynevezett \textit{seed}-et, aminek rögzítésekor megegyező módon keveri össze az adatokat. Ehhez a Python \texttt{random} modulja egyszerű megoldást biztosít.
\begin{python}
random.shuffle(self.image_names)
\end{python}

A keverés mellet véletlenszerű zajokat is hozzáadhatunk a képekhez (vágás, forgatás, elmosás). A zaj hozzáadáshoz a \texttt{imgaug} csomag rendkivül hasznos.

\begin{python}
from imgaug import augmenters as iaa

seq = iaa.Sequential([
    iaa.Crop(px=(0, 16)), # vágás 
    iaa.Fliplr(0.5), # horizontális forgatás (50%-ba) 
    iaa.GaussianBlur(sigma=(0, 3.0)) # elmosás 0-3.0 szigma-val
])
\end{python}

A \texttt{imgaug} dokumentációjában részletesen ismertetésre kerülnek az argumentumok (Flipud, Affine, SimplexNoiseAlpha, Dropout, Grayscale, Scale).

\section{A neurális háló tanítása}

\subsection{Mintaadatok beolvasása}

A tanítás elött be kell olvasni a kínai karaktereinket. Mind a tanító mind a tesztelő adathalmaznak ki kell nyerni a címkéjét, ami meghatároza, hogy a kép milyen karakter.

A különböző zajokkal ellátott képeket külön-külön be lehet tanítani a hálózattal. Ebben az esetben szét kell választani a képeket zajok szerint, majd azokat betanítani.

\subsection{Tanítás}

A kép jellemzők kiválasztásához a 4. fejezetben említett konvolúciós hálózatot használom. A CNN a jelenlegi legjobb modellek az objektumok felismerésére, a pontossága 94+\%.

\Aref{fig:arch}. ábra a hálózat architektúrát szemlélteti:

\begin{figure}[h]
	\centering
	\includegraphics[scale=0.45]{images/architecture}
	\caption{Architektúra}
	\label{fig:arch}
\end{figure}

Ez az alábbi részekből épül fel.
\begin{itemize}
\item Öt konvolúciós (convolution) réteg. Az első réteg megkapja a képet. A kép 64x64 pixel méretű és 1 szín csatorna (gray-scaled) van. A rétegeken 3x3 szűröt (kernel) használok 1-es lépéssel (stride).
\begin{python}
model.add(Convolution2D(1,	# filter rétegek száma    
                        3, 3,	# 3x3 kernel méret 
                        strides=(1,1) # lépés
                        input_shape=image))
\end{python}
\item Négy összevonó (pooling) réteg. Bemenetei a konvolúciós rétegek kimenetei. A rétegen maximális összevonás (max-pooling) van.
\begin{python}
model.add(MaxPooling2D(pool_size=(2,2)))
\end{python}
\item Két teljesen összekötött (fully-connected) réteg. Az aktivációs függvény: RELU, a dropout: 0.2, neuronok száma: 1024, 3755.
\begin{python}
model.add(Flatten()) # Bemenet
model.add(Dense(1024, activation='RELU'))
model.add(Dropout(0.2))
model.add(Dense(3755, activation='None'))
\end{python}
\end{itemize}

A tanítás a 4. fejezetben említett hiba visszaterjesztéssel (backpropagation) történik. A célunk az hogy a hibát megprobáljuk a minimumra csökkenteni, amit a hálózat súlyainak változtatásával érünk el.

\begin{python}
# Negyzetes hiba
model.compile(loss='mean_squared_error',
              optimizer='adam',
              metrics=['accuracy'])

# Halozat tanitas
model.fit_generator(generator=training_data,
                    steps_per_epoch=1000, epochs=10)
\end{python}

A tanítás ideje körülbelül 17 órát vett igénybe. A hardver: CPU - 4 mag 2.8Ghz, RAM - 4Gb. Tanítási lépések 3200epoch, tanítóhalmaz méret: 300Mb.

\section{Tesztelés}

A tesztelés során megvizsgálom, hogy mennyire érzékenyek az adott zajok a hálózatra. A tanítás fázisnál figyelembe kell venni kell venni azt, hogy ha ugyanolyan zajjal tanítjuk be a  hálózatot akkor azt csak azt fogja jól felismerni. Előállhat az overfitting.

Felismerés zajok szerint:
\begin{itemize}
\item Zaj nélküli: A kép zajjal nincs terhelve. A felismerés ebben az esetben a legmagasabb a zaj nélküli mintákra (90+\%), viszont a zajosított képeket nehezen ismeri fel.\Aref{fig:original} ábrán látható az eredti kép.

\begin{figure}[h]
	\centering
	\includegraphics[scale=1.0]{images/original}
	\caption{Eredeti kép}
	\label{fig:original}
\end{figure}

\item Pontszerű: A tanítást pontszerű zajjal terhelt képekre végezzük el. A rendszer nehezebben ismeri fel az utóbbi mintákat (75-80\%) és a zajnélküli mintákat (90+->84\%) is. A pontszerű zajok nem okoznak nagy problémát, hiszen a konvolúciós réteg az összevonó réteg megfelelő paraméterizálása megoldja a gondot.\Aref{fig:noise} ábrán látható a zajosított kép.

\begin{figure}[h]
	\centering
	\includegraphics[scale=1.0]{images/noise}
	\caption{Pontszerű zaj}
	\label{fig:noise}
\end{figure}

\item Forgatás A tanítás elforgatott képekkel. A $\pm$5-20 fokos origó körüli forgatás kis mértékben változtatja meg a hálózat felismerését. A nagyobb forgatások nagyon megnehezítik a képek felismerését (zaj nélküli: 92\%->78\%, pontszerű: 80\%->70\%). Azzal magyarázható hogy a forgatás során az alapvonások helyzete, iránya, vektorai megváltoznak. \Aref{fig:rotated} ábrán látható az elforgatott kép.

\begin{figure}[h]
	\centering
	\includegraphics[scale=1.0]{images/rotate}
	\caption{Elforgatott kép}
	\label{fig:rotated}
\end{figure}
\newpage
\item Elmosás: A tanítás homályos/elmosódott képekkel. Kis mértékben befolyásolja a hálózat felismerés aranyát. A hálózat jól alkalmazkodik a homályosított képhez.\Aref{fig:blur} ábrán látható az elmosódott kép.

\begin{figure}[h]
	\centering
	\includegraphics[scale=1.0]{images/blur}
	\caption{Elmosódott kép}
	\label{fig:blur}
\end{figure}

\end{itemize}
\Chapter{Összegzés}

A dolgozat ismerteti a kínai karakterek alapfelépítését, írási módjukat, bemutatja azok építőelemeit az alapvonásokat (strokes). Ezt követően bemutatta a vonásrendek szabályait, amely hasznos az online karakterfelismerésnél.

Részletezésre került az optikai karakterfelismerés (OCR) müködése, annak részei. Mivel a probléma már rég óta közismert, ezért áttekintésre kerültek a manapság használt OCR-es megoldások, amelyek képesek felismerni a nyomtatott kínai karaktereket különböző betűtípuson.

A minta generálás fontos eleme a bemutatott, saját karakterfelismerő rendszernek. A különböző mintákkal való betanítás az eredmények alapján a hálózatot robosztussá teszi. A különböző típusú zajok hozzáadása a képekhez, majd az azzal való tanítás növeli a hálózat általánosító képességét.

Az alapvonások kirajzolásának mechanikájának modellezése egy saját módszer kidolgozását tette szükségessé. Ennek bemutatása során a dolgozat kitért a vonal vastagságának a fontosságára. A vonal kirajzolásának dinamikáját Hermit ívek alkalmazásával sikerült megadni, majd a különféle zajok generálása és a képhez adása az OpenCV függvénykönyvtár segítségével került implementálásra.

A tématerület bemutatásához szükségesnek látszott a neurális hálózat alapvető elemeinek, tanítási és tesztelési módjának bemutatása. Ezt követően a képfelismeréshez leginkább ajánlott (az elérhető irodalmak alapján vélhetően a leghatékonyabb) konvolúciós neurális hálózatra (CNN) esett a választás. Az ezzel foglalkozó fejezet kifejti a CNN rétegeinek működését, továbbá bemutatott egy aránylag aktuális kutatási eredményt, amely a felismerés pontosságát hasonlítja össze különböző hálózat architektúrák szerint.

A validáció során betekintést nyerhetünk, az eredmények ellenőrzéséhez összeállított adathalmazba, és hogy hogyan változik a hálózat osztályozási hatékonysága a bemeneti képek és a zajjal való terhelés hatására. Összességében tehát a generált, zajjal terhelt mintákkal történő tanítási módszer javítja a konvoluciós neurális háló által adott karakterfelismerés pontosságát a zajos képek esetében.


% \Chapter{Irodalomjegyzék}
\begin{thebibliography}{9}
\addcontentsline{toc}{chapter}{Irodalomjegyzék}
% \addcontentsline{toc}{chapter}{Irodalomjegyzék}


\end{thebibliography}
 % 2

\end{document}

