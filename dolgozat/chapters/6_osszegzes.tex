\Chapter{Összegzés}

A dolgozat ismerteti a kínai karakterek alapfelépítését, írási módjukat, bemutatja azok építőelemeit az alapvonásokat (strokes). Ezt követően bemutatta a vonásrendek szabályait, amely hasznos az online karakterfelismerésnél.

Részletezésre került az optikai karakterfelismerés (OCR) müködése, annak részei. Mivel a probléma már rég óta közismert, ezért áttekintésre kerültek a manapság használt OCR-es megoldások, amelyek képesek felismerni a nyomtatott kínai karaktereket különböző betűtípuson.

A minta generálás fontos eleme a bemutatott, saját karakterfelismerő rendszernek. A különböző mintákkal való betanítás az eredmények alapján a hálózatot robosztussá teszi. A különböző típusú zajok hozzáadása a képekhez, majd az azzal való tanítás növeli a hálózat általánosító képességét.

Az alapvonások kirajzolásának mechanikájának modellezése egy saját módszer kidolgozását tette szükségessé. Ennek bemutatása során a dolgozat kitért a vonal vastagságának a fontosságára. A vonal kirajzolásának dinamikáját Hermit ívek alkalmazásával sikerült megadni, majd a különféle zajok generálása és a képhez adása az OpenCV függvénykönyvtár segítségével került implementálásra.

A tématerület bemutatásához szükségesnek látszott a neurális hálózat alapvető elemeinek, tanítási és tesztelési módjának bemutatása. Ezt követően a képfelismeréshez leginkább ajánlott (az elérhető irodalmak alapján vélhetően a leghatékonyabb) konvolúciós neurális hálózatra (CNN) esett a választás. Az ezzel foglalkozó fejezet kifejti a CNN rétegeinek működését, továbbá bemutatott egy aránylag aktuális kutatási eredményt, amely a felismerés pontosságát hasonlítja össze különböző hálózat architektúrák szerint.

A validáció során betekintést nyerhetünk, az eredmények ellenőrzéséhez összeállított adathalmazba, és hogy hogyan változik a hálózat osztályozási hatékonysága a bemeneti képek és a zajjal való terhelés hatására. Összességében tehát a generált, zajjal terhelt mintákkal történő tanítási módszer javítja a konvoluciós neurális háló által adott karakterfelismerés pontosságát a zajos képek esetében.
