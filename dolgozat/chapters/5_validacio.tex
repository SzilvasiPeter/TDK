\Chapter{Validáció}

\begin{itemize}
\item Train/Test Data
	\begin{itemize}
	\item Train -> Betanítás
	\item Test -> Valídáció	
	\end{itemize}
\end{itemize}

\begin{itemize}
\item Paraméterek változtatása
	\begin{itemize}
	\item LR
	\item Filter
	\item Layers, Activation Func
	\item Neurons, Maps
	\item ....
	\end{itemize}
\item steps, epoch, batch computing... megjelenítés matlabplot-tal
\item Learning curve (gráf és magyarázat)
\end{itemize}

\begin{itemize}
\item hagyományos neurális hálózaton
\item konvoluciós neurális hálózaton
	\begin{itemize}
	\item Architektúra 1
	\item Architektúra 2
	\item Architektúra 3
	\end{itemize}
\end{itemize}

\section{Adathalmaz}

A kézzel írott kínai karakterek sok adatbázisban megjelent, de csak az újabbak célja a nem kényszerített kézírás.


Az új CASIA-OLHWDB és CASIA-HWDB több előnnyel is rendelkeznek: nem kényszerített írás, egyidejű online és offline adatok, elkülönített minták, több kategóriák, nagyszámú írók és minták.

Az online adatkészletek biztosítják a stroke koordinátáinak sorrendjét. Az offline adatkészletek gray-scaled képek, 255 pixel értékű háttérképen.
