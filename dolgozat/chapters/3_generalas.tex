\Chapter{Minták generálása}

Tanító mintapontok előállítása

A megvalósító neurális hálózat betanítása felügyelt tanulás módszerrel történik. A minták alapján történő tanulás lényege, hogy az eljárás során a be és kimeneti mintapárokból igyekszünk megfelelő ismereteket kinyerni és ezzel a rendszer viselkedését módosítani. A hálózat feladata, hogy megtanulja a rendelkezésre álló mintapont párok által reprezentált bemenet-kimenet leképezést. Ehhez elő kell állítani a megfelelő adathalmazt.

\begin{itemize}
\item Ecset dinamika
	\begin{itemize}
	\item \(P_1 = (x_1, y_1, d_{x_1}, d_{y_1}, s_1)\) \(P_2 = (x_2, y_2, d_{x_2}, d_{y_2}, s_2)\)
	\item Ecset szín (átmenetesség)
	\end{itemize}
\item Karakter kirajzolás
	\begin{itemize}
	\item Poligonos közelítés
	\item Procedurális rajzolás
	\item Pontonkénti színszámítás (fekete-szürke árnyalat-fehér)
	\item \textit{Görbék kirajzolása (Hermit)}
	\end{itemize}
\item Ideális karakter megjelenítés
\item Ideális karakter zajosítássa
	\begin{itemize}
	\item Különböző zajok
	\item Hálózat robosztussága
	\end{itemize}
\end{itemize}

\begin{comment}{Dolgozok a Hermit-es részen.}
\end{comment}

\begin{comment}{Ezek így vázlatnak jók, viszont az egyes pontokból legalább külön-külön szakaszoknak kellene majd lenni!}
\end{comment}
