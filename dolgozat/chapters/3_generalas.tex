\Chapter{Minták generálása}

\section{Tanító mintapontok előállítása}

A megvalósító neurális hálózat betanítása felügyelt tanulás módszerrel történik. A minták alapján történő tanulás lényege, hogy az eljárás során a be és kimeneti mintapárokból igyekszünk megfelelő ismereteket kinyerni és ezzel a rendszer viselkedését módosítani. A hálózat feladata, hogy megtanulja a rendelkezésre álló mintapont párok által reprezentált bemenet-kimenet leképezést. Ehhez elő kell állítani a megfelelő adathalmazt.

Az adathalmaz előállítása elött definiálni kell a rajzolás dinamikáját. A vonal kirajzolás sorrendje mellet fontos a vonal vastagsága is. Az kézírás ugyan sokat változott, de az alapjai megmaradtak. A stroke-ok vastagsága az ecset gyorsaságától, az ecsetre ható nyomás nagyságától függ. A megfelelően rajzolt vonalakat nagyon fontosnak tartják a kínaiak, mivel a kultúrájukhoz tartozik. A kalligráfia elárulhatja az ember nemét, korát, személyességét és így tovább.

\begin{center}
\includegraphics[scale=0.8]{calligraphy}
\end{center}

\subsection{Vonal vastagság}

A következőkben részletezném a vonal vastagság változását a stroke-ok rajzolásának függvényébe. 

\begin{itemize}
\item A stroke-ok vége elvékonyul, ezt a valóságban az ecset hirtelen felemelésével érik el. (\textit{Van néhány kivétel: right fally} \includegraphics[scale=1.0]{right_fally}) 
\item Ha egy stroke végéből kezdődik egy másik stroke, akkor azok találkozásánál a vonalvastagság növekszik.
\item A horizontális vonalak közepe elvékonyul majd a végén újra vastagabb lesz. A vastagságot súlyokkal könnyedén lehet definiálni.

\begin{center}
\includegraphics[scale=0.6]{horizontal_line}
\end{center}

\item A kampós vonalaknál (hooked stroke) a "kampó" hirtelen vékonyodással és irányváltoztatással jár. Az ecset hirtelen felemelésével érik el. Általában a kampó \includegraphics[scale=1.0]{hook} stroke része.
\item A további stroke-ok általában állandó vonal vastagsággal (ecset gyorsaság állandó). A kézzel írás és festés egyén függő.
\end{itemize}

\subsection{Környezet}
\begin{itemize}
\item A kínai karakterek festése papíron történik. A mi esetünkbe ez a képernyő. A karakterek fekete(0)-fehér(255) színüek. A szürke árnyalatokat [0,255] pixel értékek reprezentálják. A képernyő teljesen fehér, de a zaj generálásánál lehet hozzáadni fekete/szürke színt. 
\begin{lstlisting}[language=Python]
img = np.zeros((512, 512, 3), np.uint8)
img[0:512] = (255, 255, 255)
\end{lstlisting}
\item A festés alapvető eszköze az ecset. Az ecset lenyomássa egy ellipszis alakot hozz létre. A festés során az ecset elfordul (megfigyelésünk szerint 45fokban). Az ellipszis területe megnő, ha a festés gyorsasága csökken, ellenkező esetben csökken. Az ellipszist könnyedén lehet rajzolni OpenCV-be:
\begin{lstlisting}
cv2.ellipse(img, center, axes, angle, start_angle, 
	end_angle, color, thickness=1, lineType=8, shift=0) 
\end{lstlisting}
...
\end{itemize}

\subsection{Karakterek kirajzolása}
\begin{enumerate}
\item Poligonos közelítés: A kínai karakter ebben az esetben pontok halmaza és az ezeket összekötő poligonok. Minnél több ponttal definiáljuk a karaktert annál jobb minőségű képet érünk el, viszont a számítás igényünk növekszik.
\item Procedurális rajzolás: Ellipsziseket rajzolunk a képernyőre a kínai karakternek megfelelően. Az ellipsziseket érintő egyenesekkel összekötjük. Az ellipszisek tengelyeinek mérete különbözhetnek egymástól.
\begin{center}
\begin{tabular}{ c c }
\includegraphics[scale=0.25]{proc_draw1} & \includegraphics[scale=0.5]{ren}
\end{tabular}
\end{center}
Az ellipszisek koordinátáit egy 5 dimenziós vektorral adhatjuk meg \(P_1 = (x_1, y_1, d_{x_1}, d_{y_1}, s_1)\) \(P_2 = (x_2, y_2, d_{x_2}, d_{y_2}, s_2)\). Az 'x' és 'y' koordináta párral az ellipszis középpontját adjuk meg. A $d_x$ és $d_y$ az ellipszis irányvektorai, amit majd a görbe kirajzolásnál lesz számunkra fontos. Az 's' az ellipszis területét adja meg. Meghatározza, hogy mennyire gyors az ecsetvonás (lassú->nagy terület, gyors->kis terület).
\begin{center}
\includegraphics[scale=0.5]{image/proc_draw2}
\end{center}
\item Pontonkénti színszámítás: A képernyőt fellehet fogni mint egy n*m méretű mátix (képernyő szélessége: n, magassága: m). Az algoritmus pixelről pixelre haladva meghatározza annak értéket (0-255). A felbontás növekedés teljesítmény romláshoz vezet.
\end{enumerate}

A három karakter kirajzolás közül a procedurális rajzolás a legeffektívebb.

\subsection{Görbék}
\textit{Ide jöhet a Hermit}

\subsection{Karakter zajosítás}
\begin{itemize}
\item Pontszerű zajok
\item Elmosódások
\item Takarás
\item Vágás
\item Forgatás
\item Gyűrődés (textúrázás)
\item Alacsony kontraszt
\item Gradiens
\end{itemize}

\textit{Line break}\\\\

\begin{itemize}
\item Ecset dinamika
	\begin{itemize}
	\item \(P_1 = (x_1, y_1, d_{x_1}, d_{y_1}, s_1)\) \(P_2 = (x_2, y_2, d_{x_2}, d_{y_2}, s_2)\)
	\item Ecset szín (átmenetesség)
	\end{itemize}
\item Karakter kirajzolás
	\begin{itemize}
	\item Poligonos közelítés
	\item Procedurális rajzolás
	\item Pontonkénti színszámítás (fekete-szürke árnyalat-fehér)
	\item \textit{Görbék kirajzolása (Hermit)}
	\end{itemize}
\item Ideális karakter megjelenítés
\item Ideális karakter zajosítássa
	\begin{itemize}
	\item Különböző zajok
	\item Hálózat robosztussága
	\end{itemize}
\end{itemize}

\begin{comment}{Dolgozok a Hermit-es részen.}
\end{comment}

\begin{comment}{Ezek így vázlatnak jók, viszont az egyes pontokból legalább külön-külön szakaszoknak kellene majd lenni!}
\end{comment}
