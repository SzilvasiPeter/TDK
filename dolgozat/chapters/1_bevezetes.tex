\Chapter{Bevezetés}

A Mesterséges Intelligencia (MI) kutatás napjaink egyik legdinamikusabban fejlődő és egyben legtöbbet vitatott tudományterülete. Sokan rajongással és határtalan elvárásokkal tekintenek rá, de legalább annyian tartanak és idegenkednek tőle. Azonban, abban mára egyetértés született, hogy az MI alapjaiban fogja átalakítani egész életünket a nem is olyan távoli jövőben. Mindez annak köszönhető, hogy az elmúlt néhány évben az MI algoritmusok hihetetlen fejlődésen mentek át, és így megvalósíthatóvá váltak a korábban csak a tudományos-fantasztikus irodalomból ismert elképzelések. A beszédfelismerés, ami ma már minden okostelefon beépített funkciója, a Google Fordító által lehetővé tett automatikus fordítás, vagy a Facebookra feltöltött fotókon látható emberek felismerése mind megoldhatatlan problémának tűnt alig egy évtizeddel ezelőtt. A dinamikus fejlődés lendületét pedig minden esetben egy új gépi tanulási szakterület, a deep learning módszerei adják.

A neurális hálózattokat az adatokból való tanulás, az optimalizálási problémák megoldásának képessége vagy az asszociatív képesség különösen alkalmassá teszi a különböző felismerési feladatok, speciálisan a képfelismerési feladatok megoldására, illetve egyéb képfeldolgozási műveletek elvégzésére.

A képfeldolgozás és alakzatfelismerés területén az úgynevezett konvolúciós neurális hálók (convolutional neural network – CNN) megjelenése hozott áttörést. A módszer a már említett arcfelismerésen túl alkalmas ábrafeliratozásra vagy például biztonsági kamerák felvételeinek automatikus feldolgozására és kiértékelésére is. A konvolúciós hálók áttörő sikere miatt mára számos alkalmazás jelent meg.

A dolgozat fő témája a kínai karakterek felismerése. A fejezetek során kitérek hogyan generálhatunk és adhatunk hozzá különböző zajokat a képekhez. Részletezem a CNN müködését és előnyeit. Továbbá bemutatom hogyan működik a tanítás és a tesztelés fázis.

\begin{comment}{Ez így kicsit általános, de elég lesz a végén átfogalmazni.}
\end{comment}
