\Chapter{Bevezetés}

A Mesterséges Intelligencia (MI) kutatás napjaink egyik legdinamikusabban fejlődő és egyben legtöbbet vitatott tudományterülete. A dinamikus fejlődés lendületét pedig minden esetben egy új gépi tanulási szakterület, a deep learning (neurális hálózatok sok réteggel) módszerei adják.

A neurális hálózattokat az adatokból való tanulás, az optimalizálási problémák megoldásának képessége vagy az asszociatív képesség különösen alkalmassá teszi a különböző felismerési feladatok, speciálisan a képfelismerési feladatok megoldására, illetve egyéb képfeldolgozási műveletek elvégzésére.

A képfeldolgozás és alakzatfelismerés területén az úgynevezett konvolúciós neurális hálók (convolutional neural network – CNN) megjelenése hozott áttörést. A módszer a már említett arcfelismerésen túl alkalmas ábrafeliratozásra vagy például biztonsági kamerák felvételeinek automatikus feldolgozására és kiértékelésére is. A konvolúciós hálók áttörő sikere miatt mára számos alkalmazás jelent meg.

A dolgozat fő témája a kínai karakterek felismerése. A fejezetek során kitérek hogyan generálhatunk és adhatunk hozzá különböző zajokat a képekhez.

Részletezem a CNN müködését és előnyeit. Feladatom során megemlítek már létező optikai karakterfelismerő (OCR) rendszereket. Kitérek azok működésére és felépítésére.

Továbbá bemutatom hogyan működik a tanítás és a tesztelés fázis. A fejezetek során leírom hogyan zajlik a tanulás a neurális hálózatokban. Matematikai képletek és illusztratív képek használatával probálom segíteni az egyszerűbb megértést.

\begin{comment}{Ez így kicsit általános, de elég lesz a végén átfogalmazni.}
\end{comment}
