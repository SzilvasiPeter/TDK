\documentclass{beamer}

\usepackage{listings}
\usepackage{color}

% Code colorature
\definecolor{paszt}{RGB}{252,252,252}
\definecolor{keret}{RGB}{220,220,220}
\lstset{
backgroundcolor=\color{paszt},
% showlines=true,
framexleftmargin=4mm,
framexrightmargin=4mm,
framextopmargin=2mm,
framexbottommargin=2mm,
frameround=tttt,
frame=trbl,
rulecolor=\color{keret}
}

\usetheme{Copenhagen}
\useinnertheme{rectangles}

% ---- Mongo theme ----
%\definecolor{light-background}{RGB}{210,250,170}
%\definecolor{dark-background}{RGB}{128,92,64}
%\usecolortheme[RGB={220,250,180}]{structure}

% ---- Vanilla theme ----
% \definecolor{light-background}{RGB}{250,250,190}
% \definecolor{dark-background}{RGB}{128,92,64}
% \usecolortheme[RGB={210, 210, 140}]{structure}

% ---- Blue theme ----
%\definecolor{light-background}{RGB}{200,220,240}
%\definecolor{dark-background}{RGB}{100,110,120}
%\usecolortheme[RGB={180, 200, 230}]{structure}

% ---- Green theme ----
\definecolor{light-background}{RGB}{229,237,204}
\definecolor{dark-background}{RGB}{156,163,140}
\usecolortheme[RGB={180, 210, 150}]{structure}

\setbeamercolor{palette primary}{fg=black, bg=light-background}
\setbeamercolor{palette quaternary}{fg=white,bg=dark-background}

\setbeamercolor{title}{fg=black}
\setbeamercolor{frametitle}{fg=black}

% Set font
\usefonttheme{structurebold}

\frenchspacing

% Language packages
\usepackage[utf8]{inputenc}
\usepackage[T1]{fontenc}
\usepackage[magyar]{babel}

% AMS
\usepackage{amssymb,amsmath}

% Graphic packages
\usepackage{graphicx}

% Syntax highlighting
% \usepackage{listings}

\usepackage{tikz}

%\begin{figure}[htb]
%\begin{center}
%	\includegraphics[scale=0.4]{ps_times.png}
%\end{center}
%\end{figure}

% ==============
\begin{document}
% ==============

\title[Kínai karakterek felismerése
generált minták alapján]{
{\Large Kínai karakterek felismerése
generált minták alapján}
}
\author[Szilvási Péter]{\Large Szilvási Péter}
\date{TDK konferencia, Miskolci Egyetem, 2018. április 25.}

% --------------------
% Title page
\frame{\titlepage}

% --------------------
\begin{frame}[fragile]
\frametitle{Kínai karakterek felismerése}

Vonások, vonás sorrend

\end{frame}

% --------------------
\begin{frame}[fragile]
\frametitle{OCR megvalósítások}

Karakterek részekre bontása, struktúrális elemzése

Betűtípusokból való eltérések

\end{frame}


% --------------------
\begin{frame}[fragile]
\frametitle{Minták generálása}


\end{frame}

% --------------------
\begin{frame}[fragile]
\frametitle{Tanítóminták zajosítása}


\end{frame}

% --------------------
\begin{frame}[fragile]
\frametitle{Mesterséges neurális hálók}

Tanítóalgoritmus, backpropagation

\end{frame}

% --------------------
\begin{frame}[fragile]
\frametitle{Konvolúciós neurális háló}

A hálózat architektúrája, használt topológia

\end{frame}

% --------------------
\begin{frame}[fragile]
\frametitle{A háló felépítése}

4.10-es ábra

\end{frame}

% --------------------
\begin{frame}[fragile]
\frametitle{A hálózat architektúrája}

5.2 ábra

\end{frame}

% --------------------
\begin{frame}[fragile]
\frametitle{Az offline adatbázis}

Kb. 3x4 karakter az 5.1 ábrából

\end{frame}

% --------------------
\begin{frame}[fragile]
\frametitle{Validáció}

Tesztelés módja, helyesség ellenőrzése

\end{frame}

% --------------------
\begin{frame}[fragile]
\frametitle{A felismerés hatékonysága}

Eredmények konkrét százalékokkal

\end{frame}

% --------------------
\begin{frame}[fragile]
\frametitle{Összegzés}


\end{frame}


% --------------------
\begin{frame}[fragile]
\frametitle{Hivatkozások}


\end{frame}

% --------------------
\begin{frame}[fragile]
    \frametitle{\ }

\begin{center}
\Large \textbf{Köszönöm szépen a figyelmet!}
\end{center}

\end{frame}


\end{document}

